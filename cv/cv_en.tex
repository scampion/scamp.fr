%%%%%%%%%%%%%%%%%%%%%%%%%%%%%%%%%%%%%%%%%
% Friggeri Resume/CV
% XeLaTeX Template
% Version 1.0 (5/5/13)
%
% This template has been downloaded from:
% http://www.LaTeXTemplates.com
%
% Original author:
% Adrien Friggeri (adrien@friggeri.net)
% https://github.com/afriggeri/CV
%
% License:
% CC BY-NC-SA 3.0 (http://creativecommons.org/licenses/by-nc-sa/3.0/)
%
% Important notes:
% This template needs to be compiled with XeLaTeX and the bibliography, if used,
% needs to be compiled with biber rather than bibtex.
%
%%%%%%%%%%%%%%%%%%%%%%%%%%%%%%%%%%%%%%%%%

\documentclass[]{friggeri-cv} % Add 'print' as an option into the square bracket to remove colors from this template for printing

%%\usepackage[backend=biber]{biblatex}

\bibliography{bibliography.bib} % Specify the bibliography file to include publications

\begin{document}


\header{Sebastien}{Campion}{Research Engineer} % Your name and current job title/field

%----------------------------------------------------------------------------------------
%	SIDEBAR SECTION
%----------------------------------------------------------------------------------------

\begin{aside} % In the aside, each new line forces a line break
\section{contact}
15, place de la Mairie
35250 Saint-Germain sur Ille
France
~
+0 (33) 6 99 40 49 23
+0 (33) 9 72 32 60 71
~
\href{mailto:sebastien.campion@gmail.com}{seb@scamp.fr}
\href{http://scamp.fr}{http://scamp.fr}
\section{languages}
french mother tongue
english \& spanish %%fluency
\section{skills}
{\color{red} $\varheartsuit$} Python, %%{\color{blue}  $\vardiamondsuit$} 
Linux
Git, Cython, C, C++, Scientific Computing, Machine Learning, Java, Javascript, PHP, HTML5
...
\end{aside}

%----------------------------------------------------------------------------------------
%	EDUCATION SECTION
%----------------------------------------------------------------------------------------

\section{education}

\begin{entrylist}
%------------------------------------------------
\entry
{1999--2002}
{Master's degree in Engineering {\normalfont of Computer Science and Telecommunications}}
{\href{https://esir.univ-rennes1.fr}{ESIR, University of Rennes 1}, France}
{Specialization in language and computer system.}
%------------------------------------------------
% \entry
% {2007--2008}
% {DEUG (two-year university degree) {\normalfont Science of Matter}}
% {University of Rouen, France}
% {Specialization in computer science}
%------------------------------------------------
\end{entrylist}

%----------------------------------------------------------------------------------------
%	WORK EXPERIENCE SECTION
%----------------------------------------------------------------------------------------

\section{experience}

\begin{entrylist}
%------------------------------------------------

\entry
{2007--Now}
{\href{http://www.inria.fr/}{INRIA}}
{Rennes, France}
{ 
Research Engineer in a public science and technology institution, I'm working mainly for the \href{http://www.irisa.fr/texmex/}{\underline{TEXMEX}} team. Our scientific topics are efficient exploitation, indexing, navigation, and access of very large multimedia databases.
 \\
TEXMEX activities : 
\begin{itemize}
\item Managing our platform for indexing  multimedia contents  
\begin{itemize}
\item Near 200TB of multimedia data
\item Large memory servers (up to 256GB of RAM)
\item 10GB/s Network
\item Local computing cluster \href{http://igrida.gforge.inria.fr}{\underline{IGRIDA}} (1200 cores) contributions
\item Digital TV and Radio capture and analyzes (MPEG2 / IP Multicast)
\end{itemize}
\item Coarse-grain distributed computing (Sun Grid Engine / \href{http://oar.imag.fr}{OAR})
\item Contributing to scientific experimentations
\item Developing and improving scientific prototypes 
\item Elaborating corpora / ground-truth for experimentation evaluations 
\item Publishing demo and online web services \url{http://texmix.irisa.fr}
\end{itemize}

National activities : 
\begin{itemize}
\item Technical director for the \href{http://gforge.inria.fr}{\underline{INRIA Forge}} 

\begin{itemize}
\item A million of hits per month
\item 14 000 registered users
\item 4 300 hosted projects
\item 1 TB of data
\item 3 x Debian/VMWare OS + NetApp NFS 
\end{itemize}

\item Trainings:
\begin{itemize}
\item \href{https://github.com/scampion/multimedia-machine-learning-tutorials}{\underline{\emph{Multimedia machine learning tutorials}}} for master degrees students
\item \emph{Software development process} for public institute lawyers managed by \href{http://www.ieepi.org}{\underline{IEEPI}} (European intellectual property institute)
\end{itemize}
\end{itemize}
}

\entry
{2002--2007}
{Orange Labs {\normalfont  (formerly France Telecom R\&D)}}
{Rennes, France}
{Junior Project Manager for TV/VOD metadata management networks
\begin{itemize}
\item Project on metadata aggregation service uses as repos for multimedia contents distributed on Orange's networks (IP, ADSL, Mobile) 
\item Mobile TV project: Specification and development of a electronic service guide for mobile terminals\cite{4114771} through an experimentation handled to DVB-H (Digital Video Broadcasting standard).
\item Development of TV/VOD portal access by Orange ADSL TV Set Top Boxes.
\end{itemize}
}

\entry
{2001}
{Thomson Grass Valley}w
{Rennes, France}
{\emph{Internship (3 months)} Development of a web monitoring tool which centralize logs generated by various audiovisual appliances (encoder, multiplexer, decoder, ...) }

\end{entrylist}

%----------------------------------------------------------------------------------------
%	INTERESTS SECTION
%----------------------------------------------------------------------------------------

\section{interests}

\textbf{professional:} digital hacks,, large scale, geopolitics, web design, software design. \\
\textbf{personal:} family life, cooking, piano, sailing.

%----------------------------------------------------------------------------------------
%	PUBLICATIONS SECTION
%----------------------------------------------------------------------------------------

\section{publications}
\printbibsection{article}{article in peer-reviewed journal} % Print all articles from the bibliography
\printbibsection{book}{books} % Print all books from the bibliography
\printbibsection{inproceedings}{international peer-reviewed conferences/proceedings} % Print all 
\printbibsection{misc}{other publications} % Print all miscellaneous entries from the bibliography
\printbibsection{report}{research reports} % Print all research reports from the bibliography
\printbibsection{patent}{patents} % Print all research reports from the bibliography

%----------------------------------------------------------------------------------------

\end{document}