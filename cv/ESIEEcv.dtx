% \iffalse
%<*never>
\documentclass{ltxdoc}

\usepackage[T1]{fontenc}
\usepackage[frenchb]{babel}
\usepackage{ESIEEcv}
\parindent=15pt

\def\fqo{<<}
\def\fqc{>>}
\AtBeginDocument{\CodelineIndex\EnableCrossrefs}
\AtEndDocument{\PrintIndex}
\begin{document}
\def\docdate{1997/12/14}
\CheckSum{213}
\DocInput{ESIEEcv.dtx}
\end{document}
%</never>
% \fi
%
% \DoNotIndex{\newline,\rightslice,\small,\itshape,\@@apports,\@@competence}
% \DoNotIndex{\@@date,\@@descr,\@@duree,\@@lieu,\@@titre,\\,\aftergroup}
% \DoNotIndex{\bigskip,\csname,\def,\edef,\else,\empty}
% \DoNotIndex{\endcsname,\endtabularx,\ensuremath,\expandafter}
% \DoNotIndex{\fi,\global,\ifx,\large,\largeurcolonne,\let}
% \DoNotIndex{\linewidth,\medskip,\newcommand,\newenvironment}
% \DoNotIndex{\newlength,\pagestyle,\par,\parindent,\phantom}
% \DoNotIndex{\RequirePackage,\scshape,\setlength,\ta,\tb}
% \DoNotIndex{\tabularx,\tb,\textbf,\the,\toksdef,\triangleright}
%
% \setcounter{tocdepth}{1}
% \GetFileInfo{ESIEEcv.sty}
%
%\iffalse \let\fqo\textquotedblleft
% \let\fqc\textquotedblright\fi
%
% \title{The \textsf{ESIEEcv}\thanks{Ce document d\'ecrit la version
% \fileversion{} du package. La documentation est dat\'ee de \docdate.} package}
% \author{Benjamin \textsc{Bayart}\\
%  Institut Gaspard \textsc{Monge}\\
%  Universit\'e de Marne-la-Vall\'ee\\
%  \texttt{bayartb@edgard.fdn.fr}
% \and
%  Pierre \textsc{Le Maguet}\\
%  Valeo Borg\\
%  Karlsruhe}
% \date{Imprim\'e le \today\\Derni\`ere mise \`a jour le \filedate}
% \maketitle
% \newcommand{\Cde}[1]{\texttt{\char92{}#1}}
%
% \begin{abstract}
% Ce document d\'ecrit l'utilisation du package \textsf{ESIEEcv} dont le but
% est de permettre la mise en page de curriculum-vit\ae{} de mani\`ere simple
% et efficace, tels que les entreprises fran\c{c}aises les attendent.
% 
% Il a \'et\'e \'ecrit par moi (Benjamin \textsc{Bayart}) sur une id\'ee
% originale (pour ce qui est de la mise en page et de la syntaxe finale) de
% Pierre \textsc{Le Maguet}.
% 
% \bigskip
%
% \noindent\emph{English}~:
% 
% This document describe the use of the package \textsf{ESIEEcv} to setup
% curriculum vit\ae{} efficiently and easily as french employers will expect it.
% 
% The current documentation is in French.
% 
% This package was written by my (Benjamin Bayart) from an original idea by
% Pierre Le Maguet (regarding design and final syntax).
% \end{abstract}
%
% \tableofcontents
%
% \clearpage
%
% \section{\fqo~The name of the game~\fqc}
%
% Ce package s'appel \textsf{ESIEEcv} simplement parce que Pierre et moi
% \'etions \'etudiants \`a l'ESIEE\footnote{\'Ecole Sup\'erieur d'Ing\'enieurs
% en \'Electronique et \'Electrotechnique.} lorsqu'il a \'et\'e \'ecrit pour
% la premi\`ere fois.
%
% \section{Organisation du document}
%
% Un cv, pour ce package, est divis\'e en plusieurs rubriques,
% classiquement 3~: \fqo~Formation initiale~\fqc, \fqo~Exp\`eriences~\fqc, et
% \fqo~Divers~\fqc.
%
% En plus des rubriques, on trouvera g\'en\'eralement au d\'ebut du cv
% un texte libre pr\'esentant la personne.
% 
% Dans chaque rubrique, on trouve des sous-rubriques permettant de
% d\'ecrire chaque information du cv. Par exemple une sous-rubrique par
% dipl\^ome dans la rubrique \fqo~formation~\fqc et une par exp\`erience
% dans la rubrique suivante.
% 
% Dans la troisi\`eme rubrique, les sous-rubrique sont g\'en\'eralement
% beaucoup plus simples, pour ce qu'elles contiennent moins
% d'information; typiquement pour chaque langue on donne juste son
% niveau, alors que pour un stage ou un ancien emploi on donne plus
% d'informations.
% 
% Les sous-rubriques les plus complexes contiendront des appels aux
% commandes \Cde{Titre}, \Cde{Date}, \Cde{Duree}, \Cde{Descr}, et \Cde{Apport}.
% Toutes ces commandes sont optionnelles (si un champ est ommi, la mise
% en page est corrig\'ee en cons\'equence), et la commande \Cde{Apport}
% peut \^etre utilis\'ee plusieurs fois pour souligner les points forts
% d'une exp\`erience ou d'une formation.
%
% Les sous-rubriques simples comportent deux champs~:
% \Cde{Descr} et \Cde{Competence}. En g\'en\'eral, on place dans la
% comp\'etence, par exemple, la langue concern\'ee, et dans la
% description le niveau que l'on a.
%
% \section{Syntaxe exacte}
%
% Un exemple de sous-rubrique simple~:
%\begin{verbatim}
%\begin{sousrubrique}
%\Competence{Espagnol}
%\Descr{Niveau scolaire}
%\end{sousrubrique}
%\end{verbatim}
%
% Un exemple de sous-rubrique complexe~:
%\begin{verbatim}
%\begin{sousrubrique}
%\Titre{Ing\'enieur ESIEE (\'Ecole Sup\'erieurs d'Ing\'enieurs
%       en \'Electronique et \'Electrotechnique)}
%\Date{1991-1996}
%\Duree{5 ans}
%\Descr{Sp\'ecialisation en informatique.}
%\Apport{Programmation syst\`eme}
%\Apport{Conception et programmation objet}
%\Apport{Th\'eorie des langages}
%\Apport{Langages interpr\'et\'es}
%\Apport{Programmation logique}
%\Apport{Programmation des interfaces graphiques}
%\Apport{SGBD}
%\end{sousrubrique}
%\end{verbatim}
%
% \emph{Attention}, si on ne souhaite donner que le lieu et pas le
% titre, il est souhaitable de faire le contraire, quitte \`a devoir
% forcer le choix de la fonte dans la donn\'ee. En effet, le
% changement de ligne qui doit avoir lieu entre le titre et la
% description n'aura effectivement lieu que si et seulement s'il y a 
% un titre \emph{et} une description.
%
% Une rubrique contiendra exclusivement une ou des sous-rubrique, et
% on devra lui passer un titre, par exemple~:
%\begin{verbatim}
%\begin{rubrique}{Formation initiale}
%\begin{sousrubrique}
%\Titre{Baccalaur\'eat...
%\end{sousrubrique}
%\end{rubrique}
%\end{verbatim}
%
% \section{Param\`etrage du syst\`eme}
%
% Un certain nombre de choix typographiques peuvent \^etre modifi\'es de
% mani\`ere simple, en red\'efinissant des commandes.
%
% La premi\`ere chose que l'on peut modifier est la structure m\^eme des
% apports. Par d\'efaut, avant chaque apport est plac\'e le symbole
% $\triangleright$ suivi d'un espace; et apr\`es chaque apport un
% point et un changement de ligne. Avant l'ensemble des apports est
% plac\'e un changement de ligne, et apr\`es l'ensemble des apports
% rien. 
%
% Ces quatre choix sont symbolis\'es par les commandes
% \Cde{PreApport}, \Cde{PostApport}, \Cde{PrePreApports} et
% \Cde{PostPostApports}. 
% 
% Pour ma part, j'aime utiliser le symbole \Cde{rightslice} du package
% \textsf{stmaryrd} et ne pas changer de ligne entre deux
% apports. Pour cela, je red\'efini les commandes \Cde{PreApport} et
% \Cde{PostApport} comme suit~:
%\begin{verbatim}
%\renewcommand{\PreApport}{\ensuremath{\rightslice} }
%\renewcommand{\PostApport}{. }
%\end{verbatim}
%
% On peut aussi, en red\'efinissant les commandes \Cde{FonteApport},
% \Cde{TailleApport}, \Cde{FonteLieu} et \Cde{FonteTitre} changer
% certains d\'etails de la mise en forme.
%
% Enfin, la colonne de gauche (celle qui contient les dates dans le
% document final) est de largeur fixe. Cette largeur (trois
% centim\`etres par d\'efaut) est d\'efinit par la longueur
% \Cde{largeurcolonne} que l'on peut ais\'ement red\'efinir, par
% exemple en faisant~:
%\begin{verbatim}
%\setlength{largeurcolonne}{2.5cm}
%\end{verbatim}
%
% \section{Le code utile \`a tous}
%
% Tout d'abord les b\^etises habituelles permettant d'identifier le
% package~:
%
%    \begin{macrocode}
%<*package>
\ProvidesPackage{ESIEEcv}[1997/12/29 v2.0a Style ESIEEcv]

\RequirePackage{tabularx}
%    \end{macrocode}
% 
% On pr\'ecise que l'on ne veut pas de num\'ero de page, ni
% d'indentation de paragraphes~:
% 
%    \begin{macrocode}
\pagestyle{empty}
\parindent=0pt
%    \end{macrocode}
%
% Enfin la d\'efinition des huit commandes utilis\'ees comme
% param\`etres par le reste du package~:
% \begin{macro}{\PreApport}
% \begin{macro}{\PostApport}
% \begin{macro}{\PrePreApports}
% \begin{macro}{\PostPostApports}
% \begin{macro}{\FonteApport}
% \begin{macro}{\TailleApport}
% \begin{macro}{\FonteLieu}
% \begin{macro}{\FonteTitre}
%    \begin{macrocode}
\newcommand{\PreApport}{\ensuremath{\triangleright} }
\newcommand{\PostApport}{.\newline}
\newcommand{\PrePreApports}{\newline}
\newcommand{\PostPostApports}{}
\newcommand{\FonteApport}{\itshape}
\newcommand{\TailleApport}{\small}
\newcommand{\FonteLieu}{\scshape}
\newcommand{\FonteTitre}{\scshape}
%    \end{macrocode}
% \end{macro}
% \end{macro}
% \end{macro}
% \end{macro}
% \end{macro}
% \end{macro}
% \end{macro}
% \end{macro}
%
% \StopEventually
%
% \section{Le code le moins utile}
%
% \subsection{D\'efinitions des longueurs et des variables de
% m\'emorisation}
%
%    \begin{macrocode}
\newlength{\Tampon}
\newlength{\Offset}
\newlength{\largeurcolonne}
\setlength{\largeurcolonne}{3cm}
\newlength{\Space}
\setlength{\Space}{3mm}
\def\@@date{}
\def\@@titre{}
\def\@@duree{}
\def\@@lieu{}
\def\@@descr{}
\def\@@apports{}
\def\@@competence{}
%    \end{macrocode}
%
% \subsection{Le syst\`eme de gestion de liste}
%
% Pour toute explication, voire annexe D du \TeX{book}.
%
% \begin{macro}{\CV@AppendItem}
%    \begin{macrocode}
\toksdef\ta=0 \toksdef\tb=2
\def\CV@AppendItem#1#2{%
 \ta={#1}%
 \tb=\expandafter{#2}%
 \global\edef#2{\the\tb\the\ta}}
%    \end{macrocode}
% \end{macro}
%
% \subsection{Initialisation du syst\`eme}
%
% \begin{macro}{\CV@ajout@apport}
% Ajoute un texte donn\'e \`a la liste des apports en cours en
% utilisant \Cde{CV@AppendItem}.
%    \begin{macrocode}
\newcommand{\CV@ajout@apport}[1]
 {\CV@AppendItem{\PreApport#1\PostApport}{\@@apports}}
%    \end{macrocode}
% \end{macro}
%
% \begin{macro}{\CV@init}
% D\'efinit la commande \Cde{\@@\textit{cde}} dont le nom est pass\'e
% en param\`etre. L'int\'er\^et est simplement d'am\'eliorer la
% compr\'ehension lors de la lecture du code~:
%    \begin{macrocode}
\newcommand{\CV@init}[1]
 {\expandafter\global\expandafter\def\csname @@#1\endcsname{}}
%    \end{macrocode}
% \end{macro}
%
% \subsection{Code de gestion des sous-rubriques}
%
% \begin{macro}{sousrubrique}
%    \begin{macrocode}
\newenvironment{sousrubrique}
%    \end{macrocode}
% On commence par initialiser toutes les variables~:
%    \begin{macrocode}
{\CV@init{apports}\CV@init{date}\CV@init{titre}\CV@init{duree}%
\CV@init{lieu}\CV@init{descr}\CV@init{competence}\CV@init{compcomment}%
%    \end{macrocode}
% Enfin on d\'efinit (localement donc) les six commandes permettant
% la m\'emorisation des donn\'ees de la sous-rubrique en cours.
%    \begin{macrocode}
\newcommand{\Date}[1]{\global\def\@@date{##1}}
\newcommand{\Duree}[1]{\global\def\@@duree{##1}}
\newcommand{\Titre}[1]{\global\def\@@titre{##1}}
\newcommand{\Lieu}[1]{\global\def\@@lieu{##1}}
\newcommand{\Descr}[1]{\global\def\@@descr{##1}}
\newcommand{\Competence}[1]{\global\def\@@competence{##1}}
\let\Apport\CV@ajout@apport
}
%    \end{macrocode}
%
% Toutes les informations collect\'ees vont maintenant \^etre mises en
% forme. Pour cela, on les balise avec les commandes \TeX{}
% appropri\'ees, puis on les m\'emorise dans une liste
% (\Cde{CV@delayed}). Le contenu de la liste sera \fqo~\'eject\'e~\fqc
% une fois l'environnement termin\'e. En effet, ces donn\'ees peuvent
% g\'en\'erer plusieurs lignes dans le tableau qui formera le cv. Ce
% qui nous force \`a ce qu'elles ne soient pas incluses dans un
% groupe. Or l'environnement est un groupe.
%
% Commen\c{c}ons pas initialiser cette liste.
%    \begin{macrocode}
{%
\global\def\CV@delayed{}
%    \end{macrocode}
%
% En premier, la date. Elle est mise en gras dans la premi\`ere
% colonne, suivie d'un changement de ligne s'il y a une dur\'ee.
% En effet, s'il y a une date et pas de dur\'ee, il faut supprimer le
% changement de ligne de la premi\`ere colonne, sans quoi la case
% correspondante du tableau serait contrainte \`a une hauteur d'au
% moins deux ligne, ce qui n'est pas acceptable.
%    \begin{macrocode}
\ifx\@@date\empty\else
 \CV@AppendItem{\textbf{\@@date}%
 \ifx\@@duree\empty\else
  \newline
 \fi
 }{\CV@delayed}%
\fi
%    \end{macrocode}
% La dur\'ee est ajout\'ee, si n\'ecessaire, entre parenth\`ese, qu'il
% y ait eu une date ou pas. On est encore dans la premi\`ere colonne.
%    \begin{macrocode}
\ifx\@@duree\empty\else
 \CV@AppendItem{(\@@duree)}{\CV@delayed}%
\fi
%    \end{macrocode}
% 
% S'il y a une comp\'etence de pr\'ecis\'ee, elle est ajout\'ee dans
% la premi\`ere colonne. On notera que le package n'impose pas que si
% on travaille avec une sous-rubrique complexe alors il n'y a pas de
% competence. De m\^eme il n'impose pas que si on est dans une
% sous-rubrique simple il doive ne pas y avoir de comp\'etence. C'est un
% choix de design~: toujours faire simple. De plus, cela permet de
% remplacer la date par n'importe quoi d'autre d'int\'eressant en ne
% mettant ni date ni dur\'ee mais simplement une comp\'etence, et ce
% m\^eme dans une sous-rubrique complexe.
%    \begin{macrocode}
\ifx\@@competence\empty\else
 \CV@AppendItem{\@@competence}{\CV@delayed}
\fi
%    \end{macrocode}
%
% Maintenant, la premi\`ere colonne est compl\`ete. On peut donc
% passer \`a la seconde. Pour ce, on ajoute un \texttt{\&}.
%    \begin{macrocode}
\CV@AppendItem{&}{\CV@delayed}
%    \end{macrocode}
%
% Le lieu et le titre sont mis en premier dans la seconde colonne,
% chacun dans la bonne fonte, et en les faisant suivre d'un point. Un
% espace est ajout\'e entre les deux si et seulement si les deux sont
% pr\'esents. 
%    \begin{macrocode}
\ifx\@@lieu\empty\else
 \CV@AppendItem{{\FonteLieu{\@@lieu}.}}{\CV@delayed}
\fi
\ifx\@@lieu\empty\else\ifx\@@titre\empty\else
 \CV@AppendItem{ }{\CV@delayed}
\fi\fi
\ifx\@@titre\empty\else
 \CV@AppendItem{{\FonteTitre{\@@titre}.}}{\CV@delayed}
\fi
%    \end{macrocode}
%
% S'il y a un titre \emph{et} une description, on ins\`ere un sauf de
% ligne dans la seconde colonne (celle de droite). Ensuite, on ajoute,
% si elle existe, la description.
%    \begin{macrocode}
\ifx\@@titre\empty\else\ifx\@@descr\empty\else
 \CV@AppendItem{\newline}{\CV@delayed}
\fi\fi
\ifx\@@descr\empty\else
 \CV@AppendItem{\@@descr}{\CV@delayed}%
\fi
%    \end{macrocode}
%
% Ne reste plus comme donn\'ee \`a ajouter que les apports dans la
% bonne fonte et dans la bonne taille. On ajoute bien entendu le
% \Cde{PrePreApports} et le \Cde{PostPostApports}.
%    \begin{macrocode}
\ifx\@@apports\empty\else
 \CV@AppendItem{{\TailleApport
                 \FonteApport
                 {\PrePreApports\@@apports\PostPostApports}}}%
               {\CV@delayed}%
\fi
%    \end{macrocode}
%
% Maintenant que toutes les donn\'ees ont \'et\'e convenablement
% balis\'ees dans la liste, ne reste plus qu'a ajouter le changement
% de ligne dans le tableau. Ensuite, on \fqo~\'ejectera~\fqc le
% contenu de la liste une fois le groupe correspondant \`a
% l'environnement termin\'e.
%    \begin{macrocode}
\CV@AppendItem{\\}{\CV@delayed}
\aftergroup\CV@delayed
}
%    \end{macrocode}
% \end{macro}
%
% \section{Code de gestion des rubriques}
%
% \begin{macro}{rubrique}
% Cet environnement est en fait extr\^emement simple. Il met son titre
% en gros et en gras, puis commence un tableau dont la largeur sera
% exactement celle de la page.
%    \begin{macrocode}
\newenvironment{rubrique}[1]{%%
\bigskip
\textbf{\large #1\phantom{Pj}} 
\medskip
\par
\tabularx{\linewidth}{p{\largeurcolonne}X}}
{\endtabularx}
%    \end{macrocode}
% \end{macro}
%
%    \begin{macrocode}
%</package>
%    \end{macrocode}
%
% \iffalse
%<*cvtest>
\documentclass[a4paper]{article}

\usepackage[T1]{fontenc}
\usepackage[frenchb]{babel}
\usepackage{ESIEEcv}

\oddsidemargin 0in
\evensidemargin 0in
\textwidth\paperwidth
\advance \textwidth by -2in
\topmargin 0in
\textheight\paperheight
\advance\textheight -2in
\headheight 0pt
\headsep 0pt
\footskip 0pt

\renewcommand{\PostApport}{. }

\begin{document}

\noindent\hspace*{\tabcolsep}\begin{minipage}{0.4\linewidth}
{\large Benjamin \textsc{Bayart}}\\
50, rue de Chamb\'ery\\
97~123 LOIN\\[3pt]
T\'el~: 09~12~11~16~10\\
Mail~: \texttt{bayartb@edgard.fdn.fr}
\end{minipage}
\begin{minipage}{0.4\linewidth}
N\'e le 24 octobre 1973 (24 ans)\\
Nationalit\'e fran\c{c}aise\\
C\'elibataire sans enfant\\
Sursitaire \`a l'incorporation
\end{minipage}


\begin{rubrique}{Formation initiale}
\begin{sousrubrique}
\Titre{Baccalaur\'eat s\'erie C (math\'ematiques et sciences
physiques)}
\Lieu{Lyc\'ee Notre-Dame Providence d'Enghien-les-Bains (95)}
\Date{1990-1991}
\end{sousrubrique}
\begin{sousrubrique}
\Titre{Ing\'enieur ESIEE (\'Ecole Sup\'erieurs d'Ing\'enieurs en \'Electronique et
\'Electrotechnique)}
\Date{1991-1996}
\Duree{5 ans}
\Descr{Sp\'ecialisation en informatique.}
\Apport{Programmation syst\`eme}
\Apport{Conception et programmation objet}
\Apport{Th\'eorie des langages}
\Apport{Langages interpr\'et\'es}
\Apport{Programmation logique}
\Apport{Programmation des interfaces graphiques}
\Apport{SGBD}
\end{sousrubrique}
\begin{sousrubrique}
\Titre{Dipl\^ome d'\'Etudes Approfondies}
\Date{1995-1996}
\Duree{1 an}
\Lieu{Universit\'e de Marne-la-Vall\'ee}
\Descr{Informatique Fondamentale et Applications}
\Apport{Th\'eorie des automates}
\Apport{Programmation logique avanc\'ee}
\Apport{Th\'eorie des partitions d'entiers}
\Apport{Calcul combinatoire}
\Apport{Algorithmique du texte}
\end{sousrubrique}
\begin{sousrubrique}
\Titre{Th\`ese de doctorat}
\Date{1996-}
\Duree{3 ans}
\Lieu{Universit\'e de Marne-la-Vall\'ee}
\Descr{Nouvelles pistes pour une typographie \'electronique de qualit\'e}
\end{sousrubrique}
\end{rubrique}

\begin{rubrique}{Exp\`eriences}
\begin{sousrubrique}
\Titre{Simulation de processeurs pour la d\'emodulation num\'erique}
\Lieu{Laboratoire d'\'Electronique Philips}
\Date{1996}
\Duree{6mois}
\Descr{\'Ecriture d'une plateforme de d\'eveloppement pour un processeur
massivement parall\`ele en developpement.}
\Apport{Projet de type industriel}
\Apport{Travail dans un d\'epartement de R\&D}
\Apport{Approche des probl\`emes d'architecture des processeurs
d\'edi\'es}
\Apport{Approche du micro-parall\`elisme}
\end{sousrubrique}
\begin{sousrubrique}
\Titre{Reconnaissance de phon\`emes par cartes de Kohonen}
\Lieu{Groupe ESIEE}
\Date{1995}
\Duree{2 mois}
\Descr{Utilisation de cartes auto-organisante de Kohonen pour la
reconnaissance et l'\'etiquettage de phon\`emes apr\`es apprentissage
non supervis\'e.}
\Apport{Travail en traitement automatis\'e du signal}
\Apport{\'Etude et utilisation des r\'eseaux de neurones}
\Apport{Premi\`ere approche de travaux de recherche}
\end{sousrubrique}
\begin{sousrubrique}
\Titre{Segmentation d'images par hyper-cartes de Kohonen}
\Lieu{Groupe ESIEE}
\Date{1995}
\Duree{6 semaines}
\Descr{Approche des techniques multi-r\'esolution.}
\end{sousrubrique}
\begin{sousrubrique}
\Titre{Gala ESIEE 94}
\Date{1994}
\Duree{1 an}
\Descr{Responsabilit\'es diverses dans l'\'equipe d'organisation d'un
grand \'ev\`enement estudiantin (6000 personnes). En particulier infographie,
communication, imprimerie, et logistique finale.}
\Apport{Travail sur la dur\'ee dans une \'equipe tr\`es soud\'ee avec
un projet directeur fort et ambitieux}
\end{sousrubrique}
\begin{sousrubrique}
\Titre{Gala ESIEE 96}
\Date{1996}
\Duree{2 mois}
\Descr{Participation \`a l'organisation finale, \`a la conception
technique de la communication, au fonctionnement de la tr\'esorerie,
et \`a la logistique finale.}
\Apport{Apprentissage de la gestion d'\'equipe et des ressources humaines}
\end{sousrubrique}
\begin{sousrubrique}
\Titre{Gala ESIEE 97}
\Date{1997}
\Duree{2 mois}
\Descr{Conseil technique du bureau d'organisation, \'etablissement de
partenariats relationels, aide \`a la gestion de la s\'ecurit\'e et de
la tr\'esorerie}
\Apport{Gestion de la motivation des personnes impliqu\'ees}
\Apport{Prise en compte de graves retards organisationels}
\end{sousrubrique}
\end{rubrique}

\begin{rubrique}{Langues et divers}
\begin{sousrubrique}
\Competence{Anglais}
\Descr{Lu, \'ecrit, parl\'e. Anglais technique courrant.}
\end{sousrubrique}
\begin{sousrubrique}
\Competence{Espagnol}
\Descr{Niveau scolaire}
\end{sousrubrique}
\begin{sousrubrique}
%\Competence{Espagnol}
\Descr{Passe-temps~: philat\'elie, typographie, gravure,
programmation, cin\'ema\dots}
\end{sousrubrique}
\end{rubrique}
\end{document}

%</cvtest>
% \fi
% \Finale
%









