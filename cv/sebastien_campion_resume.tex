%%%%%%%%%%%%%%%%%%%%%%%%%%%%%%%%%%%%%%%%%
% Friggeri Resume/CV
% XeLaTeX Template
% Version 1.0 (5/5/13)
%
% This template has been downloaded from:
% http://www.LaTeXTemplates.com
%
% Original author:
% Adrien Friggeri (adrien@friggeri.net)
% https://github.com/afriggeri/CV
%
% License:
% CC BY-NC-SA 3.0 (http://creativecommons.org/licenses/by-nc-sa/3.0/)
%
% Important notes:
% This template needs to be compiled with XeLaTeX and the bibliography, if used,
% needs to be compiled with biber rather than bibtex.
%
%%%%%%%%%%%%%%%%%%%%%%%%%%%%%%%%%%%%%%%%%

\documentclass[]{friggeri-cv} % Add 'print' as an option into the square bracket to remove colors from this template for printing

\usepackage[francais]{babel}
\usepackage{csquotes}
%%\usepackage[latin1]{inputenc}


%%\usepackage[backend=biber]{biblatex}

\bibliography{bibliography.bib} % Specify the bibliography file to include publications

\begin{document}


\header{Sébastien}{Campion}{Ingénieur de Recherche} % Your name and current job title/field

%----------------------------------------------------------------------------------------
%	SIDEBAR SECTION
%----------------------------------------------------------------------------------------

\begin{aside} % In the aside, each new line forces a line break
\section{contact}
15, place de la Mairie
35250 Saint-Germain sur Ille
France
~
+0 (33) 6 99 40 49 23
+0 (33) 9 72 32 60 71
~
\href{mailto:seb@scamp.fr}{seb@scamp.fr}
\href{http://scamp.fr}{http://scamp.fr}
\section{langues}
Français et Anglais
%english \& spanish %%fluency
\section{Compétences}
{\color{red} $\varheartsuit$} Python, %%{\color{blue}  $\vardiamondsuit$} 
Linux
Git, Cython, C, C++, Scientific Computing, Machine Learning, Java, Javascript, PHP, HTML5
...
\end{aside}

%----------------------------------------------------------------------------------------
%	EDUCATION SECTION
%----------------------------------------------------------------------------------------

\section{Formation}

\begin{entrylist}
%------------------------------------------------
\entry
{1999--2002}
{Diplôme d'Ingénieur   {\normalfont en Informatique et Télécommunications}}
{\href{https://esir.univ-rennes1.fr}{ESIR, Université de Rennes 1}, France}
{Spécialisation en langage et systèmes.}
%------------------------------------------------
% \entry
% {2007--2008}
% {DEUG (two-year university degree) {\normalfont Science of Matter}}
% {University of Rouen, France}
% {Specialization in computer science}
%------------------------------------------------
\end{entrylist}

%----------------------------------------------------------------------------------------
%	WORK EXPERIENCE SECTION
%----------------------------------------------------------------------------------------

\section{Expériences}

\begin{entrylist}

%  \entry
%  {depuis 2015}
%  {Mediego}
%  {Rennes, Bretagne, France}
%  {
%  Directeur technique \href{http://www.mediego.com}
%  }
% %------------------------------------------------
\entry
{depuis 2015}
{Mediego  \href{http://www.mediego.com}}
{Rennes, Bretagne, France}
{

Co-fondateur et Directeur technique de la société \href{https://www.mediego.com}{\underline{MEDIEGO}}

\begin{itemize}
\item Recommandation personnalisé temps réel
\item Lauréat 2015 Réseau Entreprendre, i-Lab et Creacc.
\end{itemize}

}


%------------------------------------------------
 \entry
 {2007--2015}
 {INRIA}
 {Rennes, Bretagne, France}
 {
 
 Ingénieur de recherche principalement détaché au sein de l'équipe \href{https://www-linkmedia.irisa.fr/}{\underline{LINKMEDIA}}. 
 
 Nos sujets scientifiques sont l'exploitation, l'indexation, la navigation et l'accès aux larges collections de données multimédia.
 \\
Activités principales : 

\begin{itemize}
\item Responsable de la plateforme d'indexation multimedia  (200TB de données multimédia, serveurs à large capacité de mémoire vive jusqu'a 256GB, réseau 10GB/s)
\item Contributions au meso-cluster \href{http://igrida.gforge.inria.fr}{\underline{IGRIDA}} (1200 coeurs) 
% \item Capture et analyse de flux numérique TV et Radio (MPEG2 / IP Multicast)
% \item Calcul distribué gros grains  (Sun Grid Engine / \href{http://oar.imag.fr}{OAR})
% \item Contributions aux expérimentations scientifiques
% \item Développement et amélioration de prototypes scientifiques
% \item Élaboration de corpus et vérités terrains pour les évaluations expérimentales. 
\item Publication de \href{http://texmix.irisa.fr}{\underline{démonstrateur}} et de  \href{http://allgo.irisa.fr}{\underline{services web}} en ligne.
\end{itemize}

Activités nationales : 
\begin{itemize}
\item Directeur technique de la \href{http://gforge.inria.fr}{\underline{forge INRIA}} 
(
Un million de visite par mois, 17 000 utilisateurs inscrits, 6 000 projects hébergés, 2 TB of data, 3 x Debian/VMWare OS + NetApp NFS)

\item Formations:
\begin{itemize}
\item \href{https://github.com/scampion/multimedia-machine-learning-tutorials}{\underline{\emph{Multimedia machine learning tutorials}}} pour master 2
\item \emph{Le processus de développement logiciel} à destination des chargés de valorisation des instituts publiques supervisée par l'\href{http://www.ieepi.org}{\underline{IEEPI}}
\end{itemize}
\end{itemize}
 
 }
%------------------------------------------------
\entry
{2002--2007}
 {Orange Labs {\normalfont  (France Telecom R\&D)}}
 {Rennes, Bretagne, France}
 {
 Chef de projet junior pour la gestion des métadonnées audiovisuelles sur les réseaux TV/VOD. 
 \begin{itemize}
\item Projet de service d'agrégation des métadonnées des contenus multimédias distribués sur les réseaux Orange (IP, ADSL, Mobile)
\item Projet TV Mobile: Spécification et développement d'un guide de programme électronique pour terminaux mobiles \cite{4114771} avec une expérimentation  DVB-H (Digital Video Broadcasting standard).
\item Développement  d'un portail d'accès TV/VOD  dédié aux terminaux Orange ADSL TV.
\end{itemize}
}


\entry
{2001}
{Thomson Grass Valley  {\normalfont  (Technicolor)}}
{Rennes, Bretagne, France}
{\emph{Stage (3 mois)} Développement d'un outil de monitoring web pour la visualisation et la centralisation des évènements remontés par les divers équipements (encodeurs, multiplexeurs, décodeurs, ...)}

\end{entrylist}

%----------------------------------------------------------------------------------------
%	INTERESTS SECTION
%----------------------------------------------------------------------------------------
\newpage
\section{Intérêts}

\textbf{professionnels:}  ingénierie numérique large échelle. \\
\textbf{personels:} course à pied et voile.

%----------------------------------------------------------------------------------------
%	PUBLICATIONS SECTION
%----------------------------------------------------------------------------------------

\section{Publications}
\printbibsection{article}{Article publié dans des revues} % Print all articles from the bibliography
\printbibsection{book}{books} % Print all books from the bibliography
\printbibsection{inproceedings}{Conférences internationales} % Print all 
\printbibsection{misc}{other publications} % Print all miscellaneous entries from the bibliography
\printbibsection{report}{research reports} % Print all research reports from the bibliography
\printbibsection{patent}{Brevets} % Print all research reports from the bibliography

%----------------------------------------------------------------------------------------

\end{document}